\documentclass{article}
\usepackage{amsmath}
\usepackage{amssymb}
\usepackage{titlesec}
\usepackage{amsthm}
\usepackage{breqn}
\usepackage{tikz}
\usepackage{enumitem}
\usepackage[margin=3cm]{geometry}
\setlength{\parskip}{3pt}

\title{\Huge ResLiqi: A Restaurant Reservation Liquidity System}
\date{\vspace{-5ex}} % remove the default date space
\author{\LARGE Nicolas Baez}
\begin{document}
	\maketitle
	\begin{center}
		\large Version 1.0 - February 15, 2023\\
		\vspace{25pt}
		\textbf{Abstract}
	\end{center}
		This will be written at the end
	\newpage
	\tableofcontents
	\newpage
	\section{Introduction}
	
	\hspace{1cm} Profitable restaurants have notoriously tight profit margins at 10-15\% over their net revenue, which they can only achieve if they manage inventory effectively, upsell high margin items like alcohol and coffee, and staff great people.  All while maintaining the quality of their food and competing with other restaurants on price.
	
	\hspace{1cm}This herculean effort to maintain a razor thin profit margin is undercut because restaurants are giving away one of their assets, the reservation, for free.
	
	\hspace{1cm}An 8:00pm reservation on a Saturday night is valued at the same price as a 5:00pm reservation on a tuesday, \$0.00.  While the demand for each of these seats are completely different.
	\\
	%need to include a better graph demonstrating this
	\begin{center}
	\begin{tikzpicture}
		\draw (0,0) -- (6,0);
		\draw (0,0) -- (0,4);
		\draw (1,0) -- (1,-0.2) node[below] {Tuesday};
		\draw (5,0) -- (5,-0.2) node[below] {Saturday};
		\draw (1,0) -- (1,3) node[midway, left] {Low Demand};
		\draw (5,0) -- (5,3.5) node[midway, right] {High Demand};
		\draw (3, -1) node {Price = \$0.00};
	\end{tikzpicture}
	\end{center}
	
	\hspace{1cm}Given each of these seats has completely different demand profiles, how can one currently acquire a prime-time reservation?  In one of 3 ways:
	\begin{center}
	\begin{enumerate}[label=\textbf{\Roman*}.]
		\item Reserve 2 weeks in advance at midnight when the reservations become available
		\item Use American Express or another credit cards concierge service (also well in advance)        
		\item Have a good relationship with the Maître'd or the owner
	\end{enumerate}
	\end{center}
	\hspace{1cm}After someone has acquired a prime reservation through the current system, these reservations have no liquidity.  They are given away for free and the restaurant loses all control over what happens to the reservation after.  Perhaps someone gives it away to a friend, the restaurant has no way of knowing that.  There likely exists a secondary market of "Task-Rabbit" people who are paid to stay up to get these prime-time reservations, effectively allowing the reservation to change hands multiple times at no benefit to the restaurant. %maybe delete?
	
	\hspace{1cm} Overall, restaurants are missing out on a key income opportunity by not monetizing first the \textit{initial exchange} of the reservation and the subsequent \textit{secondary exchanges} that happen all of the time.
	\newpage
	\section{Why has no-one done this before?}
	\hspace{1cm} Companies have tried to implement a reservation exchange on web 2.0 platforms like %list companies.
	The issue these companies have struggled with has been the buy in of the restaurants, and the tracking of the exchanges.  The restaurants would have to pay a consistent fee to the centralized company running the website and the servers.  Frankly, there has never been a well designed system that was done through the buy in of the restaurants, so they have all defaulted to the duopoly of OpenTable and Resy to manage their reservations.  
	\subsection{What makes ResLiqi different?}
	\hspace{1cm} ResLiqi is a system where the restaurants will earn the majority of the income generated through the reservation trading system through two income streams:
		\begin{enumerate}[label=\textbf{\Roman*}.]
			\item \textbf{Primary trading.}  A customer buying a reservation directly from the restaurant)
			\item \textbf{Secondary trading.}  A royalty fee (set by the restaurant) for each sale of the reservation on the secondary market.
		\end{enumerate}
	\subsubsection{Primary Trading Transaction Overview}
	\hspace{1cm}A primary trade is the direct sale of reservation from the restaurant to a customer for a fixed price.  In this case an 8:00pm Saturday Reservation $(R_{1})$ is exchanged for \$20.  A 2.5\% transaction fee is applied, giving the restaurant \$19.50 net.
	\begin{center}
\begin{tabular}{l c r}
	$Restaurant_{1}$& 
	$\begin{array}{c}
		\text{8:00pm Saturday Reservation } (R_{1}) \\
		\xrightarrow{\hspace*{1.5cm}} \\
		\xleftarrow{\hspace*{1.5cm}} \\
		\text{\$19.50}
			\vspace*{0.3cm}
	\end{array}$ & 
	$Customer_{1}$ \\
	\vspace*{0.3cm}
	& $\begin{array}{c}
   	\hspace*{1cm}\vspace*{0.3cm} \downarrow \text{\$0.50} \\
	Liqi\\
	\end{array}$ &
\end{tabular}
	\end{center}
	\subsubsection{Secondary Trading Transaction Overview}
	\hspace{1cm}A secondary trade is the exchange of the $R_{1}$ between customers (or traders).  The restaurant sets the royalty fee as a \% of each transaction between customers.  This can be set at 90-100\% in order to discourage or eliminate secondary trading of reservations.  In this example we set a 25\% royalty fee which is allocated directly to the restaurants wallet whenever a transaction is completed.  In this case $Customer_{1}$ sells $R_{1}$ to $Customer_{2}$ for \$40.  \$10 is allocated to the restaurant.  A 2.5\% transaction fee is applied to the transaction, but not to the royalty fee.  The royalty fee is extracted prior to the transaction fee.
	\begin{center}
\begin{tabular}{l c c r}
	$Customer_{1}$ & 
	$\begin{array}{c}
		R_{1} \\
		\xrightarrow{\hspace*{1.5cm}} \\
		\xleftarrow{\hspace*{1.5cm}} \\
		\text{\$29.25}
		\vspace*{0.3cm}
	\end{array}$ & 
	$\begin{array}{c}
	R_{1} \\
	\xrightarrow{\hspace*{1.5cm}} \\
	\xleftarrow{\hspace*{1.5cm}} \\
	\text{\$30.00}
	\vspace*{0.3cm}
\end{array}$& 
	$Customer_{2}$ \\
	&$\begin{array}{c}
		\hspace*{1cm}\vspace*{0.3cm} \downarrow \text{\$0.75} \\
		Liqi
	\end{array}$ & 
	$\begin{array}{c}
		\hspace*{1cm}\vspace*{0.3cm} \downarrow \text{\$10.00} \\
		Restaurant_{1}
	\end{array}$ 
	  &  \\
\end{tabular}

\end{center}
	\subsection{Network Equity}
		\begin{tabular}{l c c c r}
			$Customer_{1}$ & 
			$19.50$ & 
			$\xrightarrow{\hspace*{1.5cm}}\,\xleftarrow{\hspace*{1.5cm}}$ & 
			$0.50$ & 
			$Customer_{2}$ \\
			& & $\begin{array}{c}
				R_{1} \\
			\end{array}$ & & \\
		\end{tabular}
	This is the motivation.
	
	\section{Implementation}
	
	This is the methodology.
	
	\subsection{Phone Application}
	
	This is the data collection.
	
	\subsubsection{Wallet Implementation}
	
	\subsubsection{Fiat On-Ramp}
	
	\subsection{Reservation System}
	
	This is the data analysis.
	
	\section{Blockchain Backend}
	
	These are the results.
	
	\subsection{Currency}
	
	This is experiment 1.
	
	\subsection{Efficient Blockspace}
	
	\section{Auctions}
	
\end{document}
